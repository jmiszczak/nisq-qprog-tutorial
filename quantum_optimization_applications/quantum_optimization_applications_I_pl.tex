% !TeX spellcheck = pl_PL
\documentclass[a4paper,11pt]{article}
%\documentclass[lefttitle,11pt,preprint]{elsarticle}

\usepackage{fullpage}
\usepackage{hyperref}
\usepackage{graphicx}
\usepackage{subfigure}
\usepackage{polski}
\usepackage{xspace}


\newcommand{\docName}{raportu\xspace}
\newcommand{\DocName}{Report\xspace}


\begin{document}

%\begin{frontmatter}

\title{Technologie obliczeń kwantowych do rozwiązywania problemów optymalizacyjnych w rzeczywistych scenariuszach}
\author{Jaros\l aw Adam Miszczak}
\date{30/05/2023}
%\address{{Institute of Theoretical and Applied Informatics, Polish Academy
%		of Sciences}, {Baltycka 5}, {44-100}, {Gliwice}, {Poland}}
%\address{Institute of Theoretical and Applied Informatics, Polish Academy
%of Sciences,\\ Ba{\l}tycka 5, 44-100 Gliwice, Poland}
%\ead{jmiszczak@iitis.pl}
%\ead[orcid]{https://orcid.org/0000-0001-8790-101X}

\maketitle

\begin{abstract}
Możliwość wykorzystania kwantowych urządzeń obliczeniowych do osiągania prędkości nieosiągalnych dla klasycznych komputerów cyfrowych jest jedną z najważniejszych motywacji dla ostatnich postępów w technologiach kwantowych. Aby ocenić rzeczywiste możliwości technologii kwantowych, konieczne jest uwzględnienie rzeczywistych scenariuszy i porównanie wyników uzyskanych przez komputery kwantowe z klasycznymi rozwiązaniami dla realistycznych instancji. Celem niniejszego raportu jest przedstawienie przeglądu ostatnich prób wykorzystania kwantowych technologii obliczeniowych do rozwiązywania problemów optymalizacyjnych w oparciu o rzeczywiste dane. W przeciwieństwie do ogólnych przeglądów w tej dziedzinie, skupiamy się na przypadkach, w których rozwiązanie kwantowe jest opracowywane dla realistycznego zbioru danych i może zostać przełożone na rozwiązanie użyteczne w rzeczywistych zastosowaniach, a rozwiązanie uzyskane przy użyciu kwantowego podejścia obliczeniowego jest porównywane z rozwiązaniami uzyskanymi przy użyciu standardowych, klasycznych metod.
\end{abstract}

\tableofcontents

%\begin{keyword} 
%emerging technologies \sep natural computing \sep analogous computing \sep combinatorial optimization
%\end{keyword}
%
%\end{frontmatter}

\newpage

%-------------------------------------------------------------------------------
\section{Wprowadzenie}
%-------------------------------------------------------------------------------

Celem niniejszego raportu jest przedstawienie przeglądu ostatnich prób wykorzystania kwantowych technologii obliczeniowych do rozwiązywania problemów optymalizacyjnych pojawiających się w rzeczywistych scenariuszach. W tym celu skupiamy się na wybranych demonstracjach kwantowych metod rozwiązywania problemów optymalizacyjnych posiadających dwie szczególne cechy.

Po pierwsze, interesują nas głównie przypadki, w których rozwiązanie kwantowe zostało opracowane dla realistycznego zbioru danych i może zostać przełożone na rozwiązanie użyteczne w rzeczywistych zastosowaniach. Jest to poważne ograniczenie, gdyż większość badań wykorzystujących obliczenia kwantowe do rozwiązywania problemów optymalizacyjnych opiera się na modelach zabawkowych i ma na celu zademonstrowanie wykonalności podejścia opartego na obliczeniach kwantowych. Należy jednak zauważyć, że ponieważ obliczenia kwantowe zapewniają uniwersalny model obliczeń, wykorzystanie obliczeń kwantowych jest zawsze możliwe. Dlatego też w większości przypadków opracowanie wydajnych rozwiązań kwantowych wymaga dogłębnego zrozumienia obliczeń kwantowych i konkretnego problemu, który ma zostać rozwiązany za pomocą komputera kwantowego. Z tego powodu edukacja w zakresie programowania kwantowego i rozwoju oprogramowania kwantowego jest kluczowa~\cite{salehi2022computer}.

Po drugie, ponieważ jesteśmy zainteresowani faktyczną oceną użyteczności technologii kwantowych, skupiamy się na zastosowaniach, w których rozwiązanie uzyskane przy użyciu kwantowego podejścia obliczeniowego jest porównywane z rozwiązaniami uzyskanymi przy użyciu standardowych, klasycznych metod. W związku z tym dążymy do aplikacji, w których autorzy przedstawiają zarówno rozwiązanie kwantowe, jak i klasyczne, wraz z mocnymi i słabymi stronami obu podejść. Zapewnia to możliwość rzetelnego porównania podejścia kwantowego i klasycznego dla każdego przypadku.


Dwa główne podejścia do wykorzystania komputerów kwantowych w problemach optymalizacji kombinatorycznej obejmują QAOA \cite{farhi2014quantum} i QAOA+ \cite{hadfield2019quantum} dla komputerów kwantowych opartych na bramkach oraz kwantowe wyżarzanie \cite{canivell2021startup}. W obu przypadkach problem jest reprezentowany za pomocą nieograniczonej wersji modelu Isinga \cite{lucas2014ising} lub QUBO \cite{glover2018tutorial}. 

Przegląd komercyjnych zastosowań obliczeń kwantowych, identyfikujący bezpieczne szyfrowanie kwantowe, odkrywanie materiałów i leków oraz algorytmy inspirowane kwantami wśród najbardziej obiecujących zastosowań kwantowych, znajduje się w \cite{bova2021commercial}. 

Najnowszy przegląd postępów w dziedzinie kwantowego wyżarzania (QA), znanego również jako adiabatyczne obliczenia kwantowe (AQC) lub kwantowy algorytm adiabatyczny (QAA), przedstawiono w ~\cite{yarkoni2022quantum}. W odniesieniu do optymalizacji kombinatorycznej, autorzy identyfikują mobilność, harmonogramowanie i finanse jako najbardziej obiecujące obszary zastosowań kwantowych technologii obliczeniowych.



Jednym z kluczowych elementów ekosystemu obliczeń kwantowych jest oprogramowanie wspierające technologie obliczeń kwantowych. Porównanie platform oprogramowania do obliczeń kwantowych przedstawiono w ~\cite{larose2019overview}, a najnowsze przeglądy postępów i trendów w tej dziedzinie można znaleźć w \cite{zhao2020quantum}, \cite{gill2021quantum} i \cite{miszczak2023symbolic}. Ponadto, praktyczne podejście do wykorzystania optymalizacji kwantowej można znaleźć w~\cite{combarro2023practical}.

Wreszcie, należy również rozważyć ogólny przegląd kwantowych technologii obliczeniowych z perspektywy ekonomicznej i prawnej.
Badanie przedstawione w \cite{seskir2021landscape} zawiera analizę działań badawczych w dziedzinie technologii kwantowych. Dogłębną analizę niedawnej eksplozji komercyjnego zainteresowania technologiami kwantowymi można znaleźć w~\cite{seskir2022landscape}. Wreszcie, różne polityczne i prawne aspekty technologii kwantowych zostały omówione w~\cite{hoofnagle2021law}. 





Autorzy \cite{choi2019tutorial} przedstawiają dogłębną analizę algorytmu przybliżonej optymalizacji kwantowej (Quantum Approximate Optimization Algorithm - QAOA), który oferuje znaczące ulepszenia w rozwiązywaniu różnych problemów optymalizacji kombinatorycznej, a także zagłębiają się w Quantum Alternating Operator Ansatz i jego zastosowania, aby wyjaśnić koncepcję przybliżonej optymalizacji. W artykule \cite{ruan2023quantum} autorzy przedstawiają formalny opis metod kodowania dla różnych typów ograniczeń. Podają również przykłady zastosowania proponowanego schematu dla niektórych dobrze znanych problemów optymalizacyjnych.

W \cite{crosson2021prospects} przedstawiono ocenę możliwości osiągnięcia przez QA kwantowego przyspieszenie w stosunku do najlepszych klasycznych metod.

\newpage 

%-------------------------------------------------------------------------------
\section{Transport}
%-------------------------------------------------------------------------------

Będąc jednym z największych obszarów gospodarki, transport stanowi naturalny obszar do wykorzystania potencjału oferowanego przez komputery kwantowe. 

%-------------------------------------------------------------------------------
\subsection{Vehicle routing}
%-------------------------------------------------------------------------------
Jedno z pierwszych rzeczywistych zastosowań QA do optymalizacji ruchu drogowego zostało zaproponowane w ~\cite{neukart2017traffic} w ramach współpracy między Volkswagen i D-Wave Systems. W tej pracy części rzeczywistego problemu optymalizacji przepływu ruchu zostały zmapowane do postaci odpowiedniej dla urządzeń QA. Dodatkowo przeprowadzono analizę porównawczą rozwiązania kwantowego z losowym przypisaniem.

W pracy \cite{borowski2020new} wprowadzono dwa algorytmy hybrydowe i w pełni kwantowe rozwiązania problemu Capacitated Vehicle Routing Problem (CVRP) i porównano je z rozwiązaniem klasycznym. Wprowadzone algorytmy są testowane przy użyciu frameworka D-Wave na dobrze znanym zestawie wzorcowych przypadków testowych. Dodatkowo algorytmy są uruchamiane na scenariuszach testowych zbudowanych w oparciu o realistyczne sieci drogowe.


Analiza porównawcza rozwiązania kwantowego do wyznaczania tras pojazdów z Gurobi została przedstawiona w \cite{anil2022performance}. Autorzy porównali czas wymagany do znalezienia rozwiązania, a także jakość rozwiązania dla D-Wave Hybrid i Gurobi v9.5. Wyniki sugerują, że skalowanie kwantowych urządzeń wyżarzających jest lepsze w porównaniu z klasycznym podejściem.

\newpage
%-------------------------------------------------------------------------------
\subsection{Logistyka}
%-------------------------------------------------------------------------------

\cite{correll2023quantum}


%-------------------------------------------------------------------------------
\subsection{Autobusy}
%-------------------------------------------------------------------------------
 \cite{yarkoni2020quantum}

\newpage 
%-------------------------------------------------------------------------------
\subsection{Zarządzanie koleją}
%-------------------------------------------------------------------------------

Innym obszarem zastosowań, który jest badany w kontekście technologii obliczeń kwantowych, jest dyspozytorstwo i zarządzanie koleją.


In \cite{domino2022quadratic} problem zmiany harmonogramu ruchu kolejowego z powodu zakłóceń. Problem został zakodowany w formułach QUBO i HOBO i rozwiązany przy użyciu D-Wave Advanatage. Główną wadą podejścia kwantowego jest liczba kubitów dostępnych na obecnych urządzeniach, co poważnie ogranicza użyteczność proponowanych modeli. Analiza porównawcza rozwiązania opartego na D-Wave z rozwiązaniem klasycznym została przedstawiona w \cite{domino2023quantum} i wyraźnie pokazuje, że wykorzystanie obecnych urządzeń QA jest bardzo ograniczone w aspekcie zarządzania koleją w realistycznych scenariuszach. 

\newpage 

%-------------------------------------------------------------------------------
\section{Finanse}
%-------------------------------------------------------------------------------


Wykorzystanie kwantowych technologii obliczeniowych w finansach może być postrzegane jako najbardziej bezpośrednia droga do komercjalizacji ich potencjału. Można to osiągnąć poprzez wykorzystanie procedur optymalizacji obliczeń kwantowych w aplikacjach związanych z zarządzaniem pieniędzmi. Z tego powodu dostępnych jest kilka artykułów wprowadzających skierowanych do specjalistów finansowych oraz książek.

W \cite{orus2019quantum} autorzy przedstawili przegląd możliwych kwantowych podejść obliczeniowych do problemów finansowych. W szczególności przedstawiają przegląd obecnych podejść i potencjalnych perspektyw, w tym dyskusję na temat tego, w jaki sposób QA może być wykorzystywana do optymalizacji portfela, znajdowania możliwości arbitrażu i oceny zdolności kredytowej.

Wybrane zastosowania obliczeń kwantowych w finansach, koncentrujące się na optymalizacji portfela, omówiono w \cite{bouland2020prospects}. Tutaj skupiono się na znaczeniu algorytmów kwantowych dla finansów w perspektywie krótkoterminowej. Autorzy podkreślają, że niektóre z algorytmów kwantowych dla problemów finansowych powinny zapewnić przyspieszenie tylko w przypadku, gdy dostępne będą komputery kwantowe o większej skali. Dodatkowo, autorzy omawiają metody przybliżenia tych przyspieszeń do eksperymentalnej wykonalności poprzez opisanie algorytmów o niższej głębokości dla metod Monte Carlo i heurystyki QA dla optymalizacji portfela.

W~\cite{venturelli2019reverse} opracowano hybrydową kwantowo-klasyczną metodę rozwiązywania problemów optymalizacji portfela o średniej wariancji. Próbki porównawcze są w tym przypadku oparte na sparametryzowanych próbkach optymalizacji portfela w oparciu o rzeczywiste statystyki danych finansowych i zgodnie z zasadami nowoczesnej teorii portfela (MPT).


Algorytm kwantowy do analizy ryzyka kredytowego (CRA) został wprowadzony w~\cite{egger2021credit}. Autorzy zaimplementowali rozważany problem dla realistycznego rozkładu strat. Ponadto przeanalizowali skalowanie dla przypadku realistycznego rozmiaru problemu. Należy jednak zauważyć, że w tym przypadku zakłada się, że dostępne jest odporne na błędy urządzenie kwantowe. Wariant tego algorytmu przezwyciężający pewne ograniczenia oryginalnej propozycji został opisany w~\cite{dri2022towards,dri2023more}.

\newpage 

%-------------------------------------------------------------------------------
\section{Energetyka}
%-------------------------------------------------------------------------------

\cite{ajagekar2019quantum}

%\newpage 

%-------------------------------------------------------------------------------
\section{Medycyna}
%-------------------------------------------------------------------------------
%
\begin{itemize}
\item Odkrywanie leków \cite{cao2018potential}
\item  \cite{floether2023state}
%
\item \cite{cordier2022biology}
\end{itemize}
\newpage 
%-------------------------------------------------------------------------------
\section{Podsumowanie}
%-------------------------------------------------------------------------------


\paragraph{Podziękowania}
Praca została zrealizowana przy wsparciu finansowym Narodowego Centrum Nauki w ramach grantu 2019/33/B/ST6/02011 oraz Narodowego Instytutu Przetwarzania Informacji w ramach projektu POIR.04.02.00-00-D014/20-00 "Krajowa Infrastruktura Superkomputerowa dla EuroHPC -- EuroHPC PL".

\newpage
\pagestyle{empty}
\bibliographystyle{elsarticle-num}
\bibliography{quantum_optimization_applications}

\end{document}