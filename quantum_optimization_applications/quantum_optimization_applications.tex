\documentclass[a4paper,11pt]{article}
%\documentclass[lefttitle,11pt,preprint]{elsarticle}

\usepackage{fullpage}
\usepackage{hyperref}
\usepackage{graphicx}
\usepackage{subfigure}

\usepackage{xspace}


\newcommand{\docName}{report\xspace}
\newcommand{\DocName}{Report\xspace}
\providecommand{\sep}{\!\!;\xspace}

\newenvironment{keyword}{\begin{flushleft}\textbf{Keywords}:}{\!\!.\end{flushleft}}


\begin{document}

%\begin{frontmatter}

\title{Review of quantum computing technologies for solving optimization problems in real-world scenarios}
\author{Jaros\l aw Adam Miszczak}
\date{September 28, 2023}
%\address{{Institute of Theoretical and Applied Informatics, Polish Academy
%		of Sciences}, {Baltycka 5}, {44-100}, {Gliwice}, {Poland}}
%\address{Institute of Theoretical and Applied Informatics, Polish Academy
%of Sciences,\\ Ba{\l}tycka 5, 44-100 Gliwice, Poland}
%\ead{jmiszczak@iitis.pl}
%\ead[orcid]{https://orcid.org/0000-0001-8790-101X}

\maketitle

\begin{abstract}
The possibility of using quantum computing devices for delivering speed-up unachievable for classical, digital computers is one of the most important motivation for the recent progress in quantum technologies. Many attempts to use quantum computer for solving problems crucial for human civilization have been proposed and quantum computers are promised to tackled the most complex problems arising in transportation, finance, and medicine. However, to asses the actual capabilities of quantum technologies it is necessary to take into account the real-world scenarios and benchmark the results obtained by quantum computers again the classical solutions for realistic instances. The goal of this \docName is to provide an overview of the recent attempts to utilize quantum computing technologies for improving solutions of optimization problems based on real-world data or data based on the real-world scenarios. In contrast to general reviews in this area, we focus on cases where quantum solution is developed for a realistic dataset and can be translated into the solution usable in real-world applications, and where the solution obtained using quantum computing approach is benchmarked against solutions obtained using the standard methods.
\end{abstract}


\begin{keyword} 
emerging technologies \sep natural computing \sep analogous computing \sep combinatorial optimization \sep hybrid quantum-classical algorithm
\end{keyword}

\tableofcontents

%
%\end{frontmatter}

\newpage

%-------------------------------------------------------------------------------
\section{Introduction}
%-------------------------------------------------------------------------------

The goal of this \docName is it provide an overview of the recent attempts to utilize quantum computing technologies for solving optimization problems arising in real-world scenarios. To this end we focus on the selected demonstrations of quantum methods for solving optimization problems having two particular features.

First, we are mostly interested in cases where quantum solution is developed for a realistic dataset and can be translated into the solution usable in real-world applications. This is a severe limitation as most of the studies utilizing quantum computing for optimization problems are based on toy models, and are aim at demonstrating the feasibility of the approach based on quantum computing. However, one should note that, as quantum computing provides an universal model of computation, using quantum computing is always possible. Hence, in most cases developing efficient quantum solutions required a in-depth understanding of quantum computing and the particular problem to be solved using quantum computer. For this reason quantum programming and quantum software development education is crucial~\cite{salehi2022computer}.

Second, as we are interested in actually assessing the usability of quantum technologies, we focus on the application where the solution obtained using quantum computing approach is benchmarked against solutions obtained using the standard, classical methods. Hence, we aim at applications where authors provide a quantum as well as a classical solution, along with strengths and weaknesses of both approaches. This ensures that the sound comparison between the quantum and the classical approaches is possible for each case. 


Additionally, in regard to comparing the efficiency of quantum algorithms, one should also consult the recent paper \cite{lubinski2023application} where the benchmarking suite aiming to include only complete and practically useful algorithms is presented.

Two main approaches to using quantum computers in combinatorial optimization problems include QAOA \cite{farhi2014quantum} and QAOA+ \cite{hadfield2019quantum} for gate-based quntum computers, and quantum annealing \cite{canivell2021startup}. In both cases the problem is represented using an unconstrained version as Ising  \cite{lucas2014ising} or QUBO \cite{glover2018tutorial} model. 

An overview of commercial applications of quantum computing, identifying  quantum-safe encryption, material and drug discovery, and quantum-inspired algorithms among the most promising quantum applications is provided in \cite{bova2021commercial}. 

Recent overview of the progress in the area of quantum annealing (QA), also knows a adiabatic quantum computation (AQC) or Quantum Adiabatic Algorithm (QAA), is provided in~\cite{yarkoni2022quantum}. With regard to combinatorial optimization, authors identify mobility, scheduling, and finance as the most promising areas of application for quantum computing technologies.

Recent progress and trends in the field of quantum annealing, along with review of attempts to use it in different problem domains is provided in~\cite{yulianti2022implementation}.

One of the crucial element of the quantum computing ecosystem is software supporting quantum computing technologies. A comparison of quantum computing software platforms is provided in~\cite{larose2019overview} and recent surveys of the progress and trends in the field can be found in \cite{zhao2020quantum}, \cite{gill2021quantum} and \cite{miszczak2023symbolic}. Also, the hand-on approach to the utilization of quantum optimization can be found in~\cite{combarro2023practical}.

Finally, one should could also consider the high-level overview of the quantum computing technologies from the economical and legal perspective.
The study presented in \cite{seskir2021landscape} provides an analysis of the research activities in the field of quantum technologies. An in-depth analysis of the recent explosion of commercial interest in quantum technologies can be found in~\cite{seskir2022landscape}. Finally, various political and legal aspects of quantum technologies are discussed in~\cite{hoofnagle2021law}. 


Authors of \cite{choi2019tutorial} provide an in-depth exploration of the Quantum Approximate Optimization Algorithm (QAOA), which offers significant improvements in solving various combinatorial optimization problems, as well as delves into the Quantum Alternating Operator Ansatz and its applications,  to explain the concept of approximate optimization. In \cite{ruan2023quantum} authors provide a formal description of encoding methods for different types of constraints. They also provide examples of using the proposed scheme for some well-known optimization problems.

In contrast to many review papers dealign with quantum computing, we assume that the reader has some general knowledge of quantum computing. Hence, we do not provide an introduction to quantum computing.

In \cite{crosson2021prospects} the assessment of the possibility for QA to 
achieve quantum speed-up relative to classical state-of-the-art methods is provided. In \cite{ajagekar2020quantum} hybrid quantum-classical models and methods for solving large-scale mixed-integer programming problems are proposed. The proposed models are studied in molecular conformation problem, job-shop scheduling problem, manufacturing cell formation
problem, and the vehicle routing problem.



%\newpage 

%-------------------------------------------------------------------------------
\section{Transportation}
%-------------------------------------------------------------------------------

Being one of the largest areas of industry, transportation provide a natural area for harnessing the potential offered by quantum computers. In fact the possibility of using quantum computers for increasing the cost-effectiveness of various aspects of transportation is among the major incentives for some companies to invest in quantum computing technologies~\cite{bentley2022quantum, cooper2022exploring}.

%-------------------------------------------------------------------------------
\subsection{Vehicle routing}
%-------------------------------------------------------------------------------

Routing problems are studied in logistics and operations research and 
vehicle routing is a classic optimization problem that arises in many different areas, such as transportation, and logistics~\cite{dantzig1959truck,toth2002vehicle}. Routing problems encompass a wide range of optimization problems concerned with the management of a fleet of vehicles~\cite{harwood2021formulating}. 

\paragraph{Real-world traffic flow} One of the first real-word applications of QA for traffic optimization was proposed in~\cite{neukart2017traffic} as a part of collaboration between Volkswagen and D-Wave Systems. In this work parts of a real-world traffic flow optimization problem were mapper to a form suitable for QA devices. Additionally, a benchmarking of the quantum solution against the random assignment was provided.

\paragraph{Capacitated Vehicle Routing Problem} In \cite{borowski2020new} two hybrid algorithms and fully quantum solutions for Capacitated Vehicle Routing Problem (CVRP) were introduced and compared with classical solution. Introduced algorithms are benchmarked using D-Wave framework on a well-established set of benchmark test cases. Additionally, the algorithms are run on test scenarios built based on realistic road networks.

The benchmarking of quantum solution for vehicle routing against Gurobi was provide in \cite{anil2022performance}. Authors compared time required to find a solution as well as the quality of the solution for D-Wave Hybrid and  Gurobi v9.5. The results suggest that the scaling of quantum annealer devices is better comparing with the classical approach.


\paragraph{Vehicle Routing Problem with Time Windows}
Among the variants of the routing problem a particular example is given by the
vehicle routing problem with time windows (VRPTW). In \cite{leonidas2023qubit} authors formulate the VRPTW as a QUBO and apply a quantum variational approach to the VRPTW using encoding scheme described in \cite{vikstaal2020applying}. This enables them to drastically reduced the number of required qubits. The proposed approach was evaluated on a set of VRPTW instances ranging from 11 to 3964 routes constructed using data provided by researchers from ExxonMobil. The proposed algorithms was compared with the standard encoding. The comparison was conducted using simulators and cloud quantum hardware provided by IBMQ, Rigetti, and IonQ. The proposed approach gives solutions similar to the solutions found by quantum algorithms using the full encoding. Obtained solutions are compared with standard full encoding approaches for problems size  of the order of 20-30 routes, which is the maximum possible size which can be dealt with on the near-term devices.

%-------------------------------------------------------------------------------
\subsection{Logistics}
%-------------------------------------------------------------------------------

\cite{correll2023quantum}


%-------------------------------------------------------------------------------
\subsection{Buses}
%-------------------------------------------------------------------------------
 \cite{yarkoni2020quantum}

 
%-------------------------------------------------------------------------------
\subsection{Railway management}
%-------------------------------------------------------------------------------

Another area of application which is studied in the context of quantum computing technology is the railway dispatching and management. This particular example is important as the rail transport provides one of the most promising solution for developing scalable and accessible mass transport systems, which is a crucial for limit fossil fuel dependency.


In \cite{domino2022quadratic} the problem of rescheduling of railway traffic due to disruptions. The problem is encoded into QUBO and HOBO formulations, and solved using D-Wave Advanatage. The major drawback of the quantum approach is the number of qubits available on current devices, whcih severly limits the usability of the proposed models. The benchmarking of the solution based on D-Wave against the classical solution is provided in \cite{domino2023quantum} and it clearly demonstrates that the utilization of current QA devices is very limited in the aspect of railway management in realistic scenarios. 


%-------------------------------------------------------------------------------
\subsection{Space and aviation}
%-------------------------------------------------------------------------------

Recently, the Quantum Approximate Optimization Algorithms has been empleyes for the Satellite Mission Planning Problem~\cite{quetschlich2023satellite}.

%-------------------------------------------------------------------------------
\section{Finance}
%-------------------------------------------------------------------------------


Utilization of quantum computing technologies in finance can be seen as the most direct rout for commercializing their potential. This could be achieved by using quantum computing optimization routines in applications related to managing money. For this reason several introductory articles targeted at financial professionals and books are available.

In \cite{orus2019quantum} authors provided an overview of possible quantum computing approaches to financial problems. In particular, they provide an overview of current approaches and potential prospects, including the discussion of how QA can be used to portfolio optimization, arbitrage opportunities finding, and credit scoring.

Selected applications of quantum computing for finance, focusing on portfolio optimization, are discussed in \cite{bouland2020prospects}. Here, the focus is on  the relevance of quantum algorithms for finance in the near-term perspective. Authors stress that some of the some of the quantum algorithms for finance problems are expected to provide speed-ups only in the case of when larger-scale quantum computers are available. Additionally, they discuss methods for to bringing these speed-ups closer to experimental feasibility by describing lower depth algorithms for Monte Carlo methods and QA heuristics for portfolio optimization.

A hybrid quantum-classical solution method to the mean-variance portfolio optimization problems is developed in~\cite{venturelli2019reverse}. The benchmarking samples are in this case based on the parametrized samples of portfolio optimization based on real financial data statistics and following the principles of the Modern Portfolio Theory (MPT).


Quantum algorithm for Credit Risk Analysis (CRA) was introduced in~\cite{egger2021credit}. Authors implemented the considered problem for a realistic loss distribution. Moreover, they analysed the scaling  for the case of a realistic problem size. One should note, however, that in this case it is assume that the fault-tolerant quantum device is available. A variant of this algorithm overcoming some limitation of the original proposal has been described in~\cite{dri2022towards,dri2023more}.


Quantum risk analysis on IBM Q~\cite{woerner2019quantum}

\newpage 

%-------------------------------------------------------------------------------
\section{Energy}
%-------------------------------------------------------------------------------

Quantum computing is in many cases seen as one of the technologies crucial for building next generation energy production, distribution, and management systems. Authors of \cite{giani2021quantum} identify areas where quantum computing is most likely to contribute to renewable energy problems, including simulation, scheduling and dispatch, and reliability analyses. Recent research focused on utilizing quantum computing technologies for various smart grid applications is summarize in~\cite{ullah2022quantum}.


\subsection{Systems optimization}

In \cite{ajagekar2019quantum} authors discuss the potential and limitations of current quantum computers for energy system optimization. They study 
Facility allocation problem, unit commitment problem, and heat exchanger network synthesis. For all cases the compare results obtained from D-Wave 2000Q and Gurobi solver.
Discussion of prospects and challenges for using quantum computing technologies, including quantum machine learning and quantum chemistry, in the context of sustainable, renewable energy systems, are discussed in~\cite{ajagekar2022quantum}.

\emph{multi-objective configuration optimization method (COM) and energy management strategy (EMS) for ship hybrid energy system based on quantum computing} \cite{si2022configuration}
%\newpage 


\subsection{Energy management}

Recent thesis \cite{veshchezerova2022quantum} deals with optimization problems related to the charge of electric vehicles from the perspective of QAOA and quantum annealing.



%-------------------------------------------------------------------------------
\section{Medicine}
%-------------------------------------------------------------------------------
%

Next pharma disruptors \cite{cova2021artificial}

\begin{itemize}
\item Potential Drug discovery \cite{cao2018potential}
\item Requires activities and industry approach to drug discovery \cite{zinner2022institutionalization}. Check also \cite{zinner2021quantum}
\item  \cite{floether2023state}
%
\item \cite{cordier2022biology}

\item Perspective in 2022 on \cite{blunt2022perspective}
\end{itemize}
\newpage 
%-------------------------------------------------------------------------------
\section{Final remarks}
%-------------------------------------------------------------------------------


\appendix

\section{Review papers on industrial applications of quantum computing}

\begin{itemize}
\item Singh, J., Bhangu, K.S. Contemporary Quantum Computing Use Cases: Taxonomy, Review and Challenges. Arch Computat Methods Eng 30, 615–638 (2023).  \cite{singh2022contemporary} -- very broad review, many potential applications are mentioned, literature study, no in-depth analysis.

\item \cite{ullah2022quantum} -- article summarizes the research outcomes of the most recent papers, highlights their suggestions for utilizing QC techniques for various smart grid applications, and further identifies the potential smart grid applications. Several real-world QC case studies in various research fields besides power and energy systems are demonstrated.

\item Quantum Computing: Fundamentals, Trends and Perspectives for Chemical and Biochemical Engineers \cite{nourbakhsh2022quantum} -- focus on perspective for chemical and bio-chemical industry, also drug discover, literature review is also provided.

\item This paper investigates the space, fidelity and time resources of various NISQ algorithms and highlights several contradictions between NISQ requirements and actual as well as future quantum hardware capabilities. It then covers various techniques which could help like qubit fidelities improvements, various breeds of quantum error mitigation methods, analog/digital hybridization, using specific qubit types like multimode photons as well as quantum annealers and analog quantum computers (aka quantum simulators or programmable Hamiltonian simulators) which seem closer to delivering useful applications although they have their own mid to longer-term scalability challenges.  \cite{ezratty2023where}
\end{itemize}

\paragraph{Acknowledgements}
This work received support from the Polish National Science Center under the grant agreement 2019/33/B/ST6/02011 and from Polish National Information Processing Institute  under \emph{National Supercomputing Infrastructure for EuroHPC -- EuroHPC PL} project POIR.04.02.00-00-D014/20-00.

\bibliographystyle{elsarticle-num}
\bibliography{quantum_optimization_applications}

\end{document}