% !TeX spellcheck = pl_PL
\documentclass[a4paper,11pt]{article}

\usepackage{fullpage}
\usepackage{polski}
\usepackage[T1]{fontenc}
\usepackage[utf8]{inputenc}
\usepackage{times}

\usepackage{amssymb}
\usepackage{amsmath}
\usepackage{textcomp}
\usepackage{graphicx}

\usepackage{hyperref}
\hypersetup{hidelinks}
\urlstyle{same} 

\newcommand{\ang}[1]{(ang. \emph{#1})}


\begin{document}


\title{Kwantowe obliczenia wariacyjne\\ {\normalsize Rozwiązania programowe}}

\author{Jarosław Miszczak}
\date{21/12/2022}

\maketitle

\begin{abstract}
Raport przedstawia dostępne pakiety oprogramowania związane z kwantowymi algorytmami wariacyjnymi. 
\end{abstract}


%-------------------------------------------------------------------------------
\hypertarget{wprowadzenie}{%
\section{Wprowadzenie}\label{wprowadzenie}}
%-------------------------------------------------------------------------------


\newpage 

%-------------------------------------------------------------------------------
\hypertarget{sec1}{%
	\section{sec1}\label{sec1}}
%-------------------------------------------------------------------------------




\paragraph{}
\paragraph{}
\paragraph{}
\paragraph{}





%-------------------------------------------------------------------------------
\subsection{}
%-------------------------------------------------------------------------------


%-------------------------------------------------------------------------------
\subsection{}
%-------------------------------------------------------------------------------

%-------------------------------------------------------------------------------
\subsection{}
%-------------------------------------------------------------------------------




\newpage 

\hypertarget{literatura}{%
\section*{Literatura}\label{literatura}}

\begin{enumerate}
\def\labelenumi{\arabic{enumi}.}
%\tightlist

\item \url{https://pennylane.ai/}

\item \url{https://qmlsys.mit.edu/}

\item 
\end{enumerate}

\end{document}
